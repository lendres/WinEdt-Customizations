
    \documentclass{lebook}
    \usepackage{lelists}
    \setlength{\listtopsep}{0pt}
    \setlength{\leftlistindent}{0pt}
    \setlength{\rightlistindent}{0pt}

    \usepackage{lecode}
    \usepackage{legraphicextensions}
    \graphicspath{{Figures/}}

    \usepackage{lesubscripts}
    \usepackage{lehyperlink}
    \usepackage{leheadersandfooters}

    \usepackage{leaddress}
    \usepackage{times}

    \newcommand{\tbs}{$\backslash$}

	%%%%%%%%%%%%%%%%%%%%%%%%%%%%%%%%%%%%%%%%%%%%%%%%%%%%%%%%%%%%%%%%%%%%%
	% BEGIN DOCUMENT
	%%%%%%%%%%%%%%%%%%%%%%%%%%%%%%%%%%%%%%%%%%%%%%%%%%%%%%%%%%%%%%%%%%%%%

	\pagestyle{leplainheader}
	\begin{document}

    \chapter{Additional Information}
See the following directory for additional installation requirements.
    \begin{plainlist}
        \item \codetext{C:\tbs{}Custom Program Files\tbs{}LaTeX\tbs{}Processing Support\tbs{}Notes\tbs{}}
    \end{plainlist}



    \chapter{WinEdt v9 and Below}
    \section{Adding a Custom Button}

    \begin{numberedlist}
        \item Open: \codetext{Options$\rightarrow{}$Preferences$\rightarrow{}$Advanced}

        \item Edit the \codetext{WinEdt.btn} file as shown in \figurename~\ref{fig:addcustombutton}.

        \item Add a line to the end of the file similar to:
            \begin{plainlist}
                \item \codetext{204 \%B\tbs{}Bitmaps\tbs{}Buttons\tbs{}Command Prompt.bmp}
                \item \codetext{205 \%B\tbs{}Bitmaps\tbs{}Buttons\tbs{}Recycle.bmp}
                \item \codetext{206 \%B\tbs{}Bitmaps\tbs{}Buttons\tbs{}Recycle.bmp}
                \item \codetext{207 \%B\tbs{}Bitmaps\tbs{}Buttons\tbs{}image.bmp}
            \end{plainlist}
    \end{numberedlist}
The number in that line will be the button number is the toolbar setup.
Use a bmp from one previously defined.
    \begin{figure}
      \centering
      \includegraphics[width=3in]{addcustombutton}
      \caption{Edit the \codetext{WinEdt.btn}}
      \label{fig:addcustombutton}
    \end{figure}



    \section{Add a New Command}
Enter the menu setup as shown in \figurename~\ref{fig:entermenusetup}.
    \begin{figure}
        \centering
        \includegraphics[width=3in]{entermenusetup}
        \caption{Enter menu setup.}
        \label{fig:entermenusetup}
    \end{figure}

Double click an entry (as shown in \figurename~\ref{fig:doubleclickmenuitemtoopen} to open it for editing.
    \begin{figure}
        \centering
        \includegraphics[width=3in]{doubleclickmenuitemtoopen}
        \caption{Double click the menu item to open the editor for it.}
        \label{fig:doubleclickmenuitemtoopen}
    \end{figure}

Setup all the highlighted sections in \figurename~\ref{fig:commandsetup}.  The commands to use are below.
    \begin{plainlist}
        \item Make Document
        \begin{plainlist}
            \item Macro:    \codetext{Run('run\_make\_document.bat "\%N"','',0,0,'Make Document',1,1);}
            \item Start in: \codetext{\%P}
        \end{plainlist}
        \item Clean Temp Files
        \begin{plainlist}
            \item Macro:    \codetext{Run('clean\_temp\_files.bat','',0,0,'Clean Up',1,1);}
            \item Start in: \codetext{\%P}
        \end{plainlist}
        \item Clean All Output
        \begin{plainlist}
            \item Macro:    \codetext{Run('clean\_all\_output.bat','',0,0,'Clean Up',1,1);}
            \item Start in: \codetext{\%P}
        \end{plainlist}
        \item Convert All Images
        \begin{plainlist}
            \item Macro:    \codetext{Run('convert\_all\_images.bat','',0,0,'Make Images',1,1);}
            \item Start in: \codetext{\%P\tbs{}Figures\tbs{}Image Sources\tbs{}}
        \end{plainlist}
    \end{plainlist}

    \begin{figure}
        \centering
        \includegraphics[width=3in]{commandsetup}
        \caption{Ensure all the highlighted sections are set correctly.}
        \label{fig:commandsetup}
    \end{figure}


    \chapter{WinEdt v10 and Above}
    \section{Options Interface}
In version 10 and above the procedure has changed.  Every thing is accessed through the \textit{Options Interface}.  Right click on the toolbar and select the button for it (see \figurename~\ref{}
    \begin{figure}
      \centering
      \includegraphics[width=3.25in]{openingoptionsinterface}
      \caption{Opening the \textit{Options Interface} panel.}
      \label{fig:openingoptionsinterface}
    \end{figure}

    \begin{figure}
      \centering
      \includegraphics[width=2.25in]{optionsinterface}
      \caption{The \textit{Options Interface} panel.}
      \label{fig:optionsinterface}
    \end{figure}


    \section{Add Menu Items}
    \begin{numberedlist}
        \item Open the \textit{Main Menu} document from the \textit{Options Interface} panel.
        \item Search for \codetext{"TeX\_Menu"}.
        \item Under it, add the following:
    \end{numberedlist}
    \begin{plainlist}
        \item \codetext{ITEM="Make\_Document"}
        \begin{plainlist}
            \item \codetext{CAPTION="Make Document"}
            \item \codetext{IMAGE="TeXTeXif"}
            \item \codetext{SAVE\_INPUT=1}
            \item \codetext{MACRO="Run('\%P\tbs{}run\_make\_document.bat ""\%N""','\%P',0,0,'Make Document',1,1);"}
            \item \codetext{REQ\_FILTER=:"\%!M=TeX"|"\%!M=TeX:STY"|"\%!M=TeX:AUX"}
        \end{plainlist}
        \item \codetext{ITEM="Clean\_Temp\_Files"}
        \begin{plainlist}
            \item \codetext{CAPTION="Delete Temporary Files"}
            \item \codetext{IMAGE="Recycle"}
            \item \codetext{SAVE\_INPUT=1}
            \item \codetext{MACRO="Run('clean\_temp\_files.bat','\%P',0,0,'Clean Temp Files',1,1);"}
        \end{plainlist}
        \item \codetext{ITEM="Clean\_All\_Output"}
        \begin{plainlist}
            \item \codetext{CAPTION="Delete All Output Files"}
            \item \codetext{IMAGE="Recycle"}
            \item \codetext{SAVE\_INPUT=1}
            \item \codetext{MACRO="Run('clean\_all\_output.bat','\%P',0,0,'Clean All Output Files',1,1);"}
        \end{plainlist}
        \item \codetext{ITEM="Convert\_All\_Images"}
        \begin{plainlist}
            \item \codetext{CAPTION="Convert All Images"}
            \item \codetext{IMAGE="Image"}
            \item \codetext{SAVE\_INPUT=1}
            \item \codetext{MACRO="Run('convert\_all\_images.bat','\%P\tbs{}Figures\tbs{}Image Sources',0,0,'Convert All Images',1,1);"}
        \end{plainlist}
    \end{plainlist}

    \section{Add Toolbar Buttons}
    \subsection{Add \textit{Save All}}
Right click on toolbar
Find the line:
    \begin{plainlist}
        \item \codetext{BUTTON="Save"}
    \end{plainlist}

and just below it add:
    \begin{plainlist}
        \item \codetext{BUTTON="Save\_All"}
    \end{plainlist}

    \subsection{Custom Processing Buttons}
Go to the end of the main toolbar entry and add the following lines:
    \begin{plainlist}
        \item \codetext{BUTTON="|"}
        \item \codetext{BUTTON="Make\_Document"}
        \item \codetext{BUTTON="Clean\_Temp\_Files"}
        \item \codetext{BUTTON="Clean\_All\_Output"}
        \item \codetext{BUTTON="Convert\_All\_Images"}
    \end{plainlist}
    
    \subsection{Templates}
Open the file:
\codetext{\%b\tbs{}Templates\tbs{}LaTeX\tbs{}Figure.ltx}

and edit it to read:
    \begin{plainlist}
        \item \codetext{\tbs{}begin\{figure\}}
        \begin{plainlist}
            \item \codetext{\tbs{}centering}
            \item \codetext{\tbs{}includegraphics[width=]\{\}}
            \item \codetext{\tbs{}caption\{\}}
            \item \codetext{\tbs{}label\{fig:\}}
        \end{plainlist}
        \item \codetext{\tbs{}end\{figure\}}
    \end{plainlist}



	\end{document} 