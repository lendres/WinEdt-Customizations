
\documentclass{lebook}
\usepackage{lelists}
\setlength{\listtopsep}{0pt}
\setlength{\leftlistindent}{0pt}
\setlength{\rightlistindent}{0pt}

\usepackage{lefonts}
\usepackage{lefontsizes}
\usepackage{lecode}
\codefontsize{\small}

\usepackage{legraphicextensions}
\graphicspath{{Figures/}}

\usepackage{lesubscripts}
\usepackage[pdfauthor={Lance A. Endres}, pdftitle={WinEdt Installation Notes, colorlinks}]{lehyperlink}
\usepackage{leheadersandfooters}

\usepackage{leaddress}

\usepackage{times}

%%%%%%%%%%%%%%%%%%%%%%%%%%%%%%%%%%%%%%%%%%%%%%%%%%%%%%%%%%%%%%%%%%%%%
% BEGIN DOCUMENT
%%%%%%%%%%%%%%%%%%%%%%%%%%%%%%%%%%%%%%%%%%%%%%%%%%%%%%%%%%%%%%%%%%%%%

\pagestyle{leplainheader}
\begin{document}

\chapter{LeLaTeX Installation}
\begin{numberedlist}
	\item Download my LaTeX code from \url{https://github.com/lendres/LaTeX}
	\item Open the MiKTeX Console
	\item Go to \emph{Settings$\rightarrow$Directories}
	\item Add the directory of my code as a Path.
	\item Open Windows System properties
	\item Go to \emph{Advanced$\rightarrow$Environmental Variables}
	\item Add the path the \textcode{.\tbs{}Processing Support\tbs{}} directory to the path.
\end{numberedlist}


\chapter{Notes on Running}
\section{Compiling}
\begin{bulletedlist}
	\item To compile use \textcode{run\_manual\_make\_document.bat}.
\end{bulletedlist}

\section{Figures}
You can add figures in either JPG or EPS format.
\begin{numberedlist}
	\item Add the figure in the appropriate folder.
	\begin{plainlist}
		\item \textcode{.\tbs{}Figures\tbs{}Image Sources\tbs{}Eps}
		\item \textcode{.\tbs{}Figures\tbs{}Image Sources\tbs{}Jpg}
	\end{plainlist}
	\item Then run \textcode{.\tbs{}Figures\tbs{}Image Sources\tbs{}manual\_convert\_all\_images.bat}.
\end{numberedlist}



\chapter{WinEdt v9 and Below}
\section{Adding a Custom Button}

\begin{numberedlist}
	\item Open: \textcode{Options$\rightarrow{}$Preferences$\rightarrow{}$Advanced}
	
	\item Edit the \textcode{WinEdt.btn} file as shown in \figurename~\ref{fig:addcustombutton}.
	
	\item Add a line to the end of the file similar to:
	\begin{plainlist}
		\item \textcode{204 \%B\tbs{}Bitmaps\tbs{}Buttons\tbs{}Command Prompt.bmp}
		\item \textcode{205 \%B\tbs{}Bitmaps\tbs{}Buttons\tbs{}Recycle.bmp}
		\item \textcode{206 \%B\tbs{}Bitmaps\tbs{}Buttons\tbs{}Recycle.bmp}
		\item \textcode{207 \%B\tbs{}Bitmaps\tbs{}Buttons\tbs{}image.bmp}
	\end{plainlist}
\end{numberedlist}
The number in that line will be the button number is the toolbar setup.
Use a BMP from one previously defined.
\begin{figure}
	\centering
	\includegraphics[width=3in]{addcustombutton}
	\caption{Edit the \textcode{WinEdt.btn}}
	\label{fig:addcustombutton}
\end{figure}



\section{Add a New Command}
Enter the menu setup as shown in \figurename~\ref{fig:entermenusetup}.
\begin{figure}
	\centering
	\includegraphics[width=3in]{entermenusetup}
	\caption{Enter menu setup.}
	\label{fig:entermenusetup}
\end{figure}

Double click an entry (as shown in \figurename~\ref{fig:doubleclickmenuitemtoopen} to open it for editing.
\begin{figure}
	\centering
	\includegraphics[width=3in]{doubleclickmenuitemtoopen}
	\caption{Double click the menu item to open the editor for it.}
	\label{fig:doubleclickmenuitemtoopen}
\end{figure}

Setup all the highlighted sections in \figurename~\ref{fig:commandsetup}.  The commands to use are below.
\begin{plainlist}
	\item Make Document
	\begin{plainlist}
		\item Macro:    \textcode{Run('run\_make\_document.bat "\%N"','',0,0,'Make Document',1,1);}
		\item Start in: \textcode{\%P}
	\end{plainlist}
		\item Clean Temp Files
	\begin{plainlist}
		\item Macro:    \textcode{Run('clean\_temp\_files.bat','',0,0,'Clean Up',1,1);}
		\item Start in: \textcode{\%P}
	\end{plainlist}
		\item Clean All Output
	\begin{plainlist}
		\item Macro:    \textcode{Run('clean\_all\_output.bat','',0,0,'Clean Up',1,1);}
		\item Start in: \textcode{\%P}
	\end{plainlist}
		\item Convert All Images
	\begin{plainlist}
		\item Macro:    \textcode{Run('convert\_all\_images.bat','',0,0,'Make Images',1,1);}
		\item Start in: \textcode{\%P\tbs{}Figures\tbs{}Image Sources\tbs{}}
	\end{plainlist}
\end{plainlist}

\begin{figure}
    \centering
    \includegraphics[width=3in]{commandsetup}
    \caption{Ensure all the highlighted sections are set correctly.}
    \label{fig:commandsetup}
\end{figure}


\chapter{WinEdt v10 and Above}
\section{Options Interface}
In version 10 and above the procedure has changed.  Every thing is accessed through the \textit{Options Interface}.  Right click on the toolbar and select the button for it (see \figurename~\ref{fig:openingoptionsinterface}).  The \textit{Options Interface} is shown in \figurename~\ref{fig:optionsinterface}.
\begin{figure}
	\centering
	\includegraphics[width=3.25in]{openingoptionsinterface}
	\caption{Opening the \textit{Options Interface} panel.}
	\label{fig:openingoptionsinterface}
\end{figure}

\begin{figure}
	\centering
	\includegraphics[width=2.25in]{optionsinterface}
	\caption{The \textit{Options Interface} panel.}
	\label{fig:optionsinterface}
\end{figure}


\section{Add Menu Items}

\begin{numberedlist}
	\item Open the \textit{Main Menu} document from the \textit{Options Interface} panel.
       \item Open the file: \textcode{.\\src V10 and Above\\MainMenu.ini}
       \item Copy the contents from the installation notes directory file into the \textit{WinEdt} menu file.
\end{numberedlist}

\begin{bulletedlist}
	\item The new items are under the section: \textcode{"TeX\_Menu"}.
	\item They are shown below, for reference.
\end{bulletedlist}
\begin{plainlist}
	\item \textcode{ITEM="Make\_Document"}
	\begin{plainlist}
		\item \textcode{CAPTION="Make Document"}
		\item \textcode{IMAGE="TeXTeXif"}
		\item \textcode{SAVE\_INPUT=1}
		\item \textcode{MACRO="Run('\%P\tbs{}run\_make\_document.bat ""\%N""','\%P',0,0,"}
		\begin{plainlist}
			\item \textcode{'Make Document',1,1);"}
		\end{plainlist}
		\item \textcode{REQ\_FILTER=:"\%!M=TeX"|"\%!M=TeX:STY"|"\%!M=TeX:AUX"}
	\end{plainlist}
	\item \textcode{ITEM="Clean\_Temp\_Files"}
	\begin{plainlist}
		\item \textcode{CAPTION="Delete Temporary Files"}
		\item \textcode{IMAGE="Recycle"}
		\item \textcode{SAVE\_INPUT=1}
		\item \textcode{MACRO="Run('clean\_temp\_files.bat','\%P',0,0,'Clean Temp Files',1,1);"}
	\end{plainlist}
	\item \textcode{ITEM="Clean\_All\_Output"}
	\begin{plainlist}
		\item \textcode{CAPTION="Delete All Output Files"}
		\item \textcode{IMAGE="Recycle"}
		\item \textcode{SAVE\_INPUT=1}
		\item \textcode{MACRO="Run('clean\_all\_output.bat','\%P',0,0,"}
		\begin{plainlist}
			\item \textcode{'Clean All Output Files',1,1);"}
		\end{plainlist}
	\end{plainlist}
	\item \textcode{ITEM="Convert\_All\_Images"}
	\begin{plainlist}
		\item \textcode{CAPTION="Convert All Images"}
		\item \textcode{IMAGE="Image"}
		\item \textcode{SAVE\_INPUT=1}
		\item \textcode{MACRO="Run('convert\_all\_images.bat','\%P\tbs{}Figures\tbs{}Image Sources',"}
		\begin{plainlist}
			\item \textcode{0,0,'Convert All Images',1,1);"}
		\end{plainlist}
	\end{plainlist}
\end{plainlist}

\section{Add Toolbar Buttons}
\subsection{Add \textit{Save All}}
Right click on toolbar
Find the line:
\begin{plainlist}
	\item \textcode{BUTTON="Save"}
\end{plainlist}

and just below it add:
\begin{plainlist}
	\item \textcode{BUTTON="Save\_All"}
\end{plainlist}

\subsection{Custom Processing Buttons}
Go to the end of the main toolbar entry and add the following lines:
\begin{plainlist}
	\item \textcode{BUTTON="|"}
	\item \textcode{BUTTON="Make\_Document"}
	\item \textcode{BUTTON="Clean\_Temp\_Files"}
	\item \textcode{BUTTON="Clean\_All\_Output"}
	\item \textcode{BUTTON="Convert\_All\_Images"}
\end{plainlist}

\section{Templates}
   \begin{numberedlist}
       \item Open the location: \textcode{\%b\tbs{}Templates\tbs{}LaTeX\tbs{}Figure.ltx}.  Example locations are:
       \begin{plainlist}
           \item \textcode{C:\tbs{}Users\tbs{}lance\tbs{}WinEdt Team\tbs{}WinEdt 11\tbs{}Templates\tbs{}LaTeX\tbs{}Figure.ltx}
           \item \textcode{C:\tbs{}Program Files\tbs{}WinEdt Team\tbs{}WinEdt 11\tbs{}Templates\tbs{}LaTeX\tbs{}Figure.ltx}
       \end{plainlist}
       \item Overwrite the files in that directory with the ones from:
       \begin{plainlist}
           \item \textcode{C:\tbs{}Custom Program Files\tbs{}WinEdt\tbs{}src V10 and Above\tbs{}Templates}
       \end{plainlist}
   \end{numberedlist}

\section{Running Multiple Instances}
\begin{numberedlist}
	\item Select the \textcode{Additional Preferences} section of the \textcode{Application: ...} as shown in \figurename~\ref{fig:additionalpreferences}.
	\item Change the line \textcode{RUN\_ONE\_INSTANCE\_ONLY=1} to \textcode{RUN\_ONE\_INSTANCE\_ONLY=0}.
\end{numberedlist}
\begin{figure}
	\centering
	\includegraphics[width=2.5in]{additionalpreferences}
	\caption{The location of \textcode{Additional Preferences} on the \textit{Options Interface} panel.}
	\label{fig:additionalpreferences}
\end{figure}


\section{Keyboard Shortcuts}
For some keyboard shortcuts, Windows intercepts them and they do not reach WinEdt.  This is because WinEdt uses the MDI interface.  To remap toggling of tabs, download and install \textcode{PowerToys}.
\begin{plainlist}
	\item \href{https://github.com/microsoft/PowerToys}{PowerToys on Github}
\end{plainlist}

Add a keyboard shortcut to remap \textcode{Ctrl+Tab} to \textcode{Shift+F2} for \textcode{WinEdt}
Delete the \textcode{Ctrl+Tab} shortcut from the \textcode{ITEM="Next\_MDI\_Window"} entry.

\section{Tabs}
   \begin{numberedlist}
   	\item \textcode{Options$\rightarrow$Preferences$\rightarrow$Tabs}
   	\item Check \textit{Allow Keyboard Tabs}
   \end{numberedlist}

\section{User Dictionary}
\begin{numberedlist}
	\item Expand the \textit{Dictionary Manager:...} menu from the \textit{Options Interface} panel.
	\item Open the \textit{Word Lists (Dictionaries).ini} file.
	\item Find the \textcode{DICTIONARY="User (Addon)"} line.
	\item Edit the \textcode{FILE} line to point to use dictionary in the custom program files directory, e.g.:
	\begin{plainlist}
		\item FILE="C:\tbs{}Custom Program Files\tbs{}WinEdt\tbs{}Dictionary\tbs{}User.dic"
	\end{plainlist}
\end{numberedlist}

\section{Keywords and Syntax Highlighting}
\subsection{Fields}
\begin{numberedlist}
	\item Expand the \textit{Highlighting:...} menu from the \textit{Options Interface} panel.
	\item Open the \textit{Keywords.ini} file.
	\item Find the \textcode{KEYWORD\_GROUP="BibTeX Fields"} line.
	\item Add the items below into the list.
	\item Right click on the \textit{Keywords} node and select \textit{Load Script}.
\end{numberedlist}
\begin{plainlist}
	\item \textcode{abstract}
	\item \textcode{affiliation}
	\item \textcode{assignee}
	\item \textcode{day}
	\item \textcode{filedday}
	\item \textcode{filedmonth}
	\item \textcode{filedyear}
	\item \textcode{href}
	\item \textcode{id}
	\item \textcode{issuedday}
	\item \textcode{issuedmonth}
	\item \textcode{issuedyear}
	\item \textcode{logline}
	\item \textcode{pubday}
	\item \textcode{pubmonth}
	\item \textcode{pubyear}
	\item \textcode{speid}
	\item \textcode{websitename}
\end{plainlist}

\subsection{BibTeX Entry Types}
Follow the procedure similar to above, but search for the \textcode{KEYWORD\_GROUP="BibTeX"} line and add the following:
\begin{plainlist}
	\item \textcode{WEBHREF}
\end{plainlist}

\end{document} 